\documentclass{beamer}
\usetheme{metropolis}           % Use metropolis theme
\usepackage{epigraph}

\usepackage{minted}


\title{Learn You a C++ for Great Good}
\date{\today}
\author{Michael Bartling}
\institute{University of Texas at Austin\\
Department of Electrical and Computer Engineering}
\renewcommand{\epigraphflush}{center}
\renewcommand{\sourceflush}{flushleft}

\begin{document}
\maketitle

\section{C++ In a Nutshell}

\begin{frame}{Basic difference between C and C++}
    \begin{center}
        \epigraph{C makes it easy to shoot yourself in the foot; C++ makes it harder, but when you do it blows your whole leg off}{Bjarne Stroustrup\\The Creator of C++}
    \end{center}
\end{frame}

\begin{frame}[fragile]
\frametitle{3D $y = Mx + b$ in C}

\begin{minted}{c}
// Add Two 3D Vectors
Vec3d add_vec3d_vec3d(Vec3d, Vec3d);

//Multiply 3x3 Matrix with Vec3d
Vec3d mult_mat3x3_vec3d(Mat3x3, Vec3d);

// Y = Mx+b
Vec3d y = add_vec3d_vec3d( 
    mult_mat3x3_vec3d(M, x), b     
    );
\end{minted}
\end{frame}

\begin{frame}[fragile]
\frametitle{3D $y = Mx + b$ in C++}
\begin{minted}{c++}
Vec3d y = M*x + b;  //Obvious
\end{minted}

\includegraphics[width=0.95\textwidth]{ilikewhatyougot}

\end{frame}

\begin{frame}{Cool C++ Features}
    \begin{itemize}
        \item New/Delete instead of Malloc/Free
        \item Function Overloading
        \item Objects
        \item More "control" over scope
            \begin{itemize}
                \item Namespaces
                \item Membership Resolution
            \end{itemize}
        \item Template Metaprogramming
        \item Runtime Exceptions
    \end{itemize}
\end{frame}

\section{Lets Dive in to C++}

\begin{frame}[fragile]
    \frametitle{Function Overloading}
    In C, function names must be unique.
    \begin{minted}{c}
    int   add (int x, int y);
    float add (float x, float y); //Compile time error
    \end{minted}
    Developers conventionally encode the input and output types in the name. For example:
    \begin{minted}{c}
    int   add_i (int x, int y);
    float add_f (float x, float y);
    \end{minted}
    
\end{frame}

\begin{frame}[fragile]
    \frametitle{Function Overloading Cont.}

    C++ added the concept of \textbf{Name Mangling} to the compiler.
    Functions with the same "names" but different parameters get assigned unique identifiers at compile time!
\end{frame}    
\begin{frame}[fragile]
    \frametitle{Function Overloading Cont.}
    For example g++ will compile the following two functions

    \begin{minted}{c++}
    int   add (int x, int y);
    float add (float x, float y); 
    \end{minted}
    And the output will look like this (note: using the \texttt{nm} command)
    \begin{minted}{c++}
    __Z3addff
    __Z3addii
    \end{minted}
    Here the \_\_Z is a reserved label for the compiler (so we don't need to care about it). 
    The 3 means the function "name" is the next 3 characters. Finally, the \texttt{ff} and \texttt{ii} indicate two float and two int input parameters respectively.

\end{frame}


\begin{frame}[fragile]
    \frametitle{Objects}
    C style structs are really just "Plain Old Data" (POD).
    And any API using these structs operates on some reference to an object.

    \begin{minted}{c}
    typedef struct Michael {
        int  age;
        bool isAwesome;
    } Michael;
    Michael* constructMichael(){ 
        Michael* m = (Michael*)malloc(sizeof(Michael)); 
        m->age = 24; m->isAwesome = true; return m}
    void destroyMichael(Michael* thisMichael)
        { free(thisMichael); }
    void isMichaelAwesome(Michael* thisMichael)
        { printf("true"); //Assumed }
    \end{minted}


\end{frame}

\begin{frame}[fragile]
    \frametitle{Basic Objects in C++}
    C++ Allows you to embed all this crap directly "in" the object. The compiler will create functions with an implicit pointer to the current object (this).

    \begin{minted}{c++}
    class Michael{
      private:
        int  _age;
        bool _isAwesome; //only Michael can modify this
      public:
        Michael(void) { _age = 24; _isAwesome = true; }
        ~Michael(void) { /*Destructor not necessary*/ }
        bool isAwesome(void){ return _isAwesome; }
    };
    \end{minted}
\end{frame}

\begin{frame}[fragile]
    \frametitle{Basic Objects in C++}
    The previous C++ class is roughly equivalent to to following.
    \begin{minted}{c}
    struct Michael{
        int  _age;
        bool _isAwesome; 
    };
    Michael createMichael() 
    { Michael m; m._age = 24; m._isAwesome = true; }
    void destroyMichael(Michael& this) 
    { /*Destructor not necessary*/ }
    bool isAwesome(Michael& m){ return m._isAwesome; }
    \end{minted}
\end{frame}

\begin{frame}[fragile]
    \frametitle{Basic Objects in C++}

    However, since the \texttt{\_age} and \texttt{\_isAwesome} are private to Michael the following is invalid.

    \begin{minted}{c++}
    Michael m;
    m._isAwesome = false;   // compile time error
    cout << m._age << endl; // compile time error
    \end{minted}

    Only members of the Michael class have access to private fields, however everyone has access to the public space.
    \begin{minted}{c++}
    Michael m;
    cout << m.isAwesome() << endl; //Valid and Assumed   
    \end{minted}
\end{frame}

\begin{frame}
    \frametitle{Object Lifecycles}
    Conceptually, the object lifecycle is pretty simple.
    \begin{enumerate}
        \item Programmer creates the object by calling some constructor
        \item Programmer uses object
        \item Object destructor called automatically when object is out of scope or if manually deleted via \texttt{delete}.
    \end{enumerate}
\end{frame}

\begin{frame}
    \frametitle{Object Lifecycles: Constructors}
    Constructors literally construct types. 
    For example, a constructor might initialize variables, set some sort of state, allocate space on the heap, or even copy objects.

    Constructors are functions with the same name as the class. A few special case constructors:
    \begin{itemize}
    \item Default constructor has no input parameters
    \item Copy constructors take some reference to an object of the same type
    \item Move constructors (noteworthy, but not important for this class :) )
    \end{itemize}


\end{frame}

\begin{frame}[fragile]
    \frametitle{Object Lifecycles: Constructors}
    Imagine we have a Pokemon class:
    \begin{minted}{c++}
    Pokemon() //Default Constructor
     { type = nullptr; moveSet.append(Tackle(pp=25)); 
       health = 10; ...}
    
    Pokemon(string mType, int mHealth, ...)
     { type = new PokeType(mType); ... }
    //Pass by const lvalue reference 
    Pokemon(const Pokemon& that) // Copy Constructor
     { copy(that); return this; }
    \end{minted}
\end{frame}

\begin{frame}
\frametitle{Object Lifecycles: Destructors}
    Destructors literally destroy objects, and by default they do nothing.
    Destructors are only necessary to help clean up after the object. 
    For example, use destructors to:
    \begin{itemize}
    \item delete (free) any memory dynamically allocated by the class
    \item notify engine of object destruction (Advanced and rarely used)
    \end{itemize}

    \textbf{Destructors are called when objects go out of scope (automatically added by compiler) or when they are deleted.}
    Think about why this is awesome (hint: containers of objects).
\end{frame}

\begin{frame}[fragile]
    \frametitle{Object Lifecycles: Destructors}
    So for our Pokemon example, we need a destructor to free up the Pokemon type data.

    \begin{minted}{c++}
    ~Pokemon(){ 
        if(type) 
            delete type;
        ...
    }
    \end{minted}
\end{frame}

\begin{frame}[fragile]
    \frametitle{Object Lifecycles: Example 1}
    \begin{minted}{c++}
    #include <iostream>
    #include <string>
    using namespace std;
    class Human{ //private by default
      string name;
      public:
        Human(string name): name(name) 
        { cout <<"Constructing: " << name << endl; }
        ~Human(){ cout << "RIP: " << name << endl; }
    };
    int main(){
        Human m = Human("Ash Ketchum");
        cout << "Hello" << endl;
        return 0;
    }
    \end{minted}
\end{frame}
\begin{frame}[fragile]
    \frametitle{Object Lifecycles: Example 1}
    This Outputs
    \begin{minted}{c++}
    Constructing: Ash Ketchum
    Hello
    RIP: Ash Ketchum
    \end{minted}

\end{frame}

\begin{frame}[fragile]
    \frametitle{Object Lifecycles: Example 2}
    \begin{minted}{c++}
    void foo(){ Human m("Jim-Bob-Joe"); }

    int main(){
        Human m = Human("Ash Ketchum");
        foo();
        cout << "Hello" << endl;
        return 0;
    }
    \end{minted}
\end{frame}
\begin{frame}[fragile]
    \frametitle{Object Lifecycles: Example 2}
    This Outputs
    \begin{minted}{c++}
    Constructing: Ash Ketchum
    Constructing: Jim-Bob-Joe
    RIP: Jim-Bob-Joe
    Hello
    RIP: Ash Ketchum
    \end{minted}

\end{frame}


\begin{frame}[fragile]
    \frametitle{Object Lifecycles: Example 3}
    \begin{minted}{c++}

    int main(){
        Human* m =  new Human("Ash Ketchum");
        delete m;
        cout << "Hello" << endl;
        return 0;
    }
    \end{minted}
\end{frame}
\begin{frame}[fragile]
    \frametitle{Object Lifecycles: Example 3}
    This Outputs
    \begin{minted}{c++}
    Constructing: Ash Ketchum
    RIP: Ash Ketchum
    Hello
    \end{minted}

\end{frame}

\begin{frame}[fragile]
    \frametitle{Object Operator Overloading }
    You can also define operators the same way you define functions!
    \begin{minted}{c++}
    Vec2d operator + (Vec2d a, Vec2d b) 
     { return Vec2d(a.x + b.x, a.y + b.y); }
    Vec2d operator + (Vec2d a, double b)
     { return Vec2d(a.x + b, a.y + b); }
    Vec2d operator + (double b, Vec2d a)
     { return Vec2d(a.x + b, a.y + b); }
    \end{minted}

\end{frame}

\begin{frame}[fragile]
    \frametitle{Namespaces and Scope Resolution}
    \begin{itemize}
    \item Namespaces == named scope/block of code
        \begin{itemize}
        \item Useful for big projects/libraries
        \end{itemize}
    \item Scope resolution operator, \texttt{Scope::Qualified\_Name}
    \end{itemize}
    \begin{minted}{c++}
    namespace Vallath{ 
      int magicMathFunction(...); //Probably has bugs 
    }
    namespace Michael{ 
      int magicMathFunction(...); //Always Perfect 
    }
    ...
    //Use the function in the Michael namespace
    int byteMe = Michael::magicMathFunction(...);
    \end{minted}
\end{frame}
\begin{frame}[fragile]
    \frametitle{Namespaces and Scope Resolution}
    Scope resolution super useful for embedded types
    \begin{minted}{c++}
    class Vector{
        ...
        class iterator{...};
        iterator begin(void){...}
        iterator end(void){...}
    };
    ...
    // Iterate through the Vector elements
    for(Vector::iterator i = mVec.begin();
        i != mVec.end(); i++){
            ...
    }
    \end{minted}
\end{frame}

\begin{frame}
\frametitle{Basic Takeaways}
    \begin{itemize}
            \item Proper use of constructors/destructors makes it easier to debug memory leaks
            \item Objects encapsulate both logic and variables meaning code is smaller, is cleaner, and, generally, does more with less
            \item Language basics help expose more bugs at compile time
            \item C++ super scalable as compared to C
            \item Despite being more generic and reusable, many of the C++ STL datastructures and algorithms outperform hand tuned C code!
    \end{itemize}
\end{frame}
\begin{frame}
\frametitle{Basic Takeaways}
    \includegraphics[width=0.95\textwidth]{shutup.jpg}
\end{frame}
\begin{frame}
\frametitle{Templates: Make Magic by Exploiting Types}

\end{frame}
\end{document}
